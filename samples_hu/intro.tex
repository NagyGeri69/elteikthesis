\chapter{Bevezetés}
\label{ch:intro}

A digitális technológia fejlődésével és az internet széles körű elterjedésével a klasszikus társasjátékok is kezdenek új formát ölteni. A hagyományos, asztali játékok digitális adaptációi nem csupán megőrzik az eredeti játékok szellemiségét, hanem új lehetőségekkel is gazdagítják azokat. A digitalizáció lehetővé teszi, hogy kedvenc játékainkat térben és időben egymástól távol lévő játékosok is élvezhessék, valamint hogy a mesterséges intelligencia segítségével egyedül is gyakorolhassunk és fejlődhessünk.

A kalaha, más néven mancala, az egyik legősibb táblajáték, amelynek eredete több ezer évre nyúlik vissza. Ez a stratégiai játék Afrikából származik, és az évszázadok során számos változata alakult ki. A játék egyszerű szabályai mögött komplex stratégiai döntések húzódnak meg, ami ideálissá teszi mind kezdő, mind haladó játékosok számára. A kalaha különlegessége, hogy míg szabályai könnyen elsajátíthatók, a győztes stratégiák kidolgozása és alkalmazása jelentős gyakorlást és gondolkodást igényel.

\section{Témaválasztás indoklása}

A témaválasztást több tényező is motiválta. Elsősorban a klasszikus társasjátékok digitalizálásának növekvő igénye, hiszen a modern életvitel mellett egyre kevesebb lehetőség nyílik a személyes találkozásra és játékra. Egy online platform lehetővé teszi, hogy a játékosok akkor is játszhassanak egymással, amikor fizikailag nem tudnak egy asztalhoz ülni. Másrészt a kalaha játék matematikai és stratégiai mélysége kiváló alapot nyújt egy mesterséges intelligencia alapú ellenféljátékos megvalósításához, ami új kihívást jelenthet a fejlesztésben.

A Unity játékmotor választása mellett szólt annak széleskörű támogatottsága, fejlett hálózati képességei és a platformfüggetlen fejlesztés lehetősége. A C\# programozási nyelv használata pedig lehetővé teszi a tiszta, jól strukturált és könnyen karbantartható kód írását.

\section{Célkitűzések}

A szakdolgozat célja egy olyan modern, felhasználóbarát kalaha játék implementálása, amely:
\begin{itemize}
	\item Lehetővé teszi két játékos számára az online játékot
	\item Tartalmaz különböző nehézségi szintű mesterséges intelligencia ellenfeleket
	\item Biztosítja a játék szabályainak pontos betartását
	\item Kellemes felhasználói élményt nyújt intuitív kezelőfelülettel
	\item Stabil és megbízható hálózati kapcsolatot biztosít
	\item Vizuálisan vonzó és reszponzív grafikus felülettel rendelkezik
\end{itemize}

\section{A dolgozat felépítése}

A dolgozat a következő főbb fejezetekre tagolódik. A felhasználói dokumentáció részletesen bemutatja a program telepítését és használatát, kitérve minden játékmódra és funkcióra. A fejlesztői dokumentáció ismerteti a program architektúráját, a főbb tervezési döntéseket és azok indoklását, valamint részletesen tárgyalja az online játék és a mesterséges intelligencia megvalósítását. A dokumentáció kitér a tesztelési folyamatokra és eredményekre is. Végül az összefoglalásban értékeljük az elért eredményeket és felvázoljuk a lehetséges továbbfejlesztési irányokat.