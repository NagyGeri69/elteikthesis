\chapter{Fejlesztői dokumentáció}
\label{ch:impl}
\section{Tervezési fázis}
\subsection{Követelmények elemzése}
A tervezési fázis első lépéseként azonosítottam a kulcsfontosságú követelményeket:
\begin{itemize}
	\item A játéknak támogatnia kell három különböző játékmódot: helyi többjátékos, online többjátékos és AI elleni játék
	\item A játékmenetnek gördülékenynek kell lennie, vizuálisan vonzó animációkkal
	\item Az AI-nak skálázható nehézségi szintekkel kell rendelkeznie
	\item Az online játéknak megbízhatóan kell működnie különböző hálózati körülmények között
	\item A felhasználói felületnek intuitívnak és reszponzívnak kell lennie
\end{itemize}
\subsection{Architektúrális tervezés}
A követelmények alapján egy MVC (Model-View-Controller) architektúra mellett döntöttem, amely jól illeszkedik a Unity játékmotor működéséhez és a projekt igényeihez:
\begin{itemize}
	\item \textbf{Model:} A játék üzleti logikája és adatmodellje (BoardManager, GameManager)
	\begin{itemize}
		\item Tábla állapotának kezelése
		\item Játékszabályok implementálása
		\item Lépés validációja
	\end{itemize}
	\item \textbf{View:} Unity scene-ek, és UI elemek (StoneManager)
	\begin{itemize}
		\item 3D játéktábla modell megjelenítése
		\item Felhasználói felület elemei (gombok, szövegek)
		\item Vizuális visszajelzések és animációk
	\end{itemize}
	
	\item \textbf{Controller:} Játékosok és játékmenet vezérlése
	\begin{itemize}
		\item Különböző játékostípusok (HumanPlayer, AIPlayer, NetworkPlayer)
		\item Bemenet kezelése
		\item Játékállapot-változások koordinálása
	\end{itemize}
\end{itemize}
Az architektúra erősen eseményvezérelt, ami lehetővé teszi a komponensek laza csatolását és a játék állapotváltozásainak hatékony kezelését. Például amikor egy játékos lép, az eseményrendszer biztosítja, hogy minden érintett komponens (animációk, UI elemek, hálózati szinkronizáció) megfelelően reagáljon.
\subsection{Technológiai stack kiválasztása}
A technológiák kiválasztásánál több szempontot is figyelembe vettem:
\begin{itemize}
	\item \textbf{Unity játékmotor:} A döntés mellett szólt:
	\begin{itemize}
		\item Modern fizikai motor, ami elengedhetetlen a kövek mozgásának realisztikus szimulációjához
		\item Erős közösségi támogatás és dokumentáció
		\item Beépített networking megoldások
		\item Platformfüggetlen fejlesztés lehetősége
		\item Van már tapasztalatom a használatában
	\end{itemize}
	\item \textbf{Unity Netcode for GameObjects:} Az online többjátékos mód implementálásához:
	\begin{itemize}
		\item Host-Client modell támogatása
		\item Automatikus állapotszinkronizáció
		\item Beépített biztonsági megoldások
	\end{itemize}
	
	\item \textbf{Unity Relay:} A játékosok közötti kapcsolat biztosításához:
	\begin{itemize}
		\item NAT átjárás problémák kezelése
		\item Biztonságos kapcsolat
		\item Alacsony látencia
	\end{itemize}
	
	\item \textbf{Python az AI implementációhoz:} A döntés okai:
	\begin{itemize}
		\item Gyors prototípus készítés lehetősége
		\item Hatékony algoritmus implementáció
		\item Könnyű debuggolás és tesztelés
		\item Gazdag matematikai és algoritmus könyvtárak
	\end{itemize}
\end{itemize}
\section{Rendszerarchitektúra}
\subsection{Főbb komponensek és felelősségeik}
\subsubsection{GameManager (Model)}
A játék központi vezérlő komponense, amely:
\begin{itemize}
	\item Kezeli a különböző játékállapotokat
	\item Koordinálja a játékmódok közötti váltást
	\item Menedzseli a játékosok létrehozását és inicializálását
	\item Eseményeket közvetít a komponensek között
\end{itemize}
\begin{lstlisting}[language={[Sharp]C}]
	public class GameManager : NetworkBehaviour
	{
		public void StartAIGame(int strength)
		{
			currentGameMode = GameMode.VsAI;
			currentGameState = GameState.Running;
			player1Object = new GameObject();
			HumanPlayer player1 = player1Object.AddComponent<HumanPlayer>();
			player1.Id = PlayerIdGenerator.GetNextId();
			
			player2Object = new GameObject();
			AIPlayer aiPlayer = player2Object.AddComponent<AIPlayer>();
			aiPlayer.strength = strength;
			aiPlayer.Id = PlayerIdGenerator.GetNextId();
			
			this.player1 = player1;
			this.player2 = aiPlayer;        
			
			SceneManager.sceneLoaded += AIGameLoaded;
			SceneManager.LoadScene("GameScene");
		}
	}
\end{lstlisting}
\subsubsection{BoardManager (Model)}
A játék logikai motorja, amely:
\begin{itemize}
	\item Tárolja és kezeli a játéktábla állapotát
	\item Validálja a játékosok lépéseit
	\item Ellenőrzi a játékszabályok betartását
	\item Eseményeket generál a játékállapot változásairól
\end{itemize}
\subsubsection{StoneManager (View)}
A játék vizuális megjelenítéséért felelős komponens:
\begin{itemize}
	\item Kezeli a kövek fizikai reprezentációját
	\item Animálja a kövek mozgását
	\item Szinkronizálja a vizuális állapotot a játéklogikával
	\item Események alapján frissíti a megjelenítést
\end{itemize}
\section{AI implementáció}
Az AI modul a minimax algoritmus egy optimalizált implementációját használja alpha-beta vágással. Az implementáció Python nyelven történt, ami lehetővé tette a gyors prototípuskészítést és hatékony tesztelést.
\subsection{Az algoritmus működése}
A minimax algoritmus rekurzívan építi fel a lehetséges lépések fáját:
\begin{itemize}
	\item A fa minden csomópontja egy játékállást reprezentál
	\item A levelek értékelése a játékos szempontjából történik
	\item Az alpha-beta vágás jelentősen csökkenti az átvizsgálandó csomópontok számát
	\item A keresési mélység dinamikusan változik az AI szintjétől függően
\end{itemize}
A döntéshozatal folyamata:
\begin{enumerate}
	\item Az aktuális játékállás kiértékelése
	\item Lehetséges lépések generálása
	\item Minden lépésre rekurzív kiértékelés
	\item A legjobb értékelésű lépés kiválasztása
\end{enumerate}
\subsection{Nehézségi szintek implementációja}
A különböző AI szintek megvalósítása a keresési mélység és a kiértékelési függvény módosításával történik:
\begin{lstlisting}[language=Python]
	def calculate_kalaha_move(board_state, is_player_1, ai_level):
	
	depth = min(ai_level + 1, 6)  # Base depth from AI level
	
	eval_score, selected_move = minimax(
		board_state, 
		depth,
		float('-inf'),
		float('inf'),
		is_player_1,
		True
	)
	
	return selected_move
\end{lstlisting}
\section{Tesztelés és teljesítménymérés}
\subsection{AI teljesítmény elemzése}
Az AI különböző nehézségi szintjeinek tesztelése egy dedikált tesztkörnyezetben történt. A tesztek során az AI random stratégiát alkalmazó ellenfelek ellen játszott. Minden nehézségi szinten 100 játszmát futattunk le, az eredmények:
\begin{itemize}
	\item 1-es szint: 59% nyerési arány (59/100 játszma)
	\item 2-es szint: 73% nyerési arány (73/100 játszma)
	\item 3-as szint: 77% nyerési arány (77/100 játszma)
	\item 4-es szint: 82% nyerési arány (82/100 játszma)
	\item 5-ös szint: 90% nyerési arány (90/100 játszma)
\end{itemize}
Az eredmények alapján látható, hogy:
\begin{itemize}
	\item Az AI már az 1-es szinten is képes a random stratégiánál jobb teljesítményre
	\item A nehézségi szintek között jelentős teljesítménykülönbség van
	\item Az 5-ös szintű AI már dominálja a random lépéseket tevő ügynököt
\end{itemize}
A teljesítménynövekedés magyarázata:
\begin{itemize}
	\item Magasabb nehézségi szinteken mélyebb keresés
\end{itemize}
\section{További fejlesztési lehetőségek}
\begin{itemize}
	\item Az AI továbbfejlesztése gépi tanulási módszerekkel
	\item Játékmenet rögzítése és visszajátszási lehetőség
	\item Részletes statisztikai rendszer és ranglisták
	\item Több játékos egyidejű megfigyelési lehetősége
	\item Mobil platformra való portolás
\end{itemize}
\end{parameter>
	</invoke>