\chapter{Fejlesztői dokumentáció}
\label{ch:impl}

\section{Architektúra és tervezési döntések}

\subsection{Technológiai stack}
A játék fejlesztéséhez az alábbi technológiákat választottam:

\begin{itemize}
	\item \textbf{Unity játékmotor:} A Unity választását több tényező indokolta:
	\begin{itemize}
		\item Beépített fizikai motor a kövek mozgásának szimulálásához
		\item Fejlett hálózati képességek az online játékhoz
		\item Hatékony 3D renderelés
		\item Platformfüggetlen fejlesztés lehetősége
	\end{itemize}
	\item \textbf{C\# programozási nyelv:} A Unity natív nyelve, amely modern és tiszta kód írását teszi lehetővé
	\item \textbf{Unity Netcode for GameObjects:} Az online többjátékos funkciók megvalósításához
	\item \textbf{Unity Relay:} A játékosok közötti közvetlen kapcsolat biztosításához
\end{itemize}

\subsection{Architektúrális döntések}
A program architektúrája a Manager alapú tervezési mintát követi, ahol minden főbb funkcionalitásért egy dedikált Manager osztály felel. Ez az architektúra több előnnyel is jár:

\begin{itemize}
	\item Tiszta felelősségi körök
	\item Könnyű bővíthetőség
	\item Jól tesztelhető komponensek
	\item Egyszerű függőség kezelés
\end{itemize}

\subsection{Főbb komponensek}

\subsubsection{GameManager}
A GameManager a játék központi vezérlő komponense, amely:
\begin{itemize}
	\item Kezeli a játék állapotait (lobby, játék, szünet, játék vége)
	\item Koordinálja a többi manager működését
	\item Biztosítja a játékmódok közötti váltást
	\item Kezeli a játékosok csatlakozását és leválását
\end{itemize}

\begin{lstlisting}[language={[Sharp]C}]
	public class GameManager : NetworkBehaviour
	{
		public static GameManager Instance;
		private GameMode currentGameMode;
		public GameState currentGameState { get; set; }
		
		private IPlayerController player1;
		private IPlayerController player2;
		
		public void StartOnlineGame(string sceneName, LoadSceneMode loadSceneMode, 
		List<ulong> clientsCompleted, List<ulong> clientsTimedOut)
		{
			currentGameMode = GameMode.Online;
			currentGameState = GameState.Running;
			bool shouldFlip = UnityEngine.Random.value > 0.5f;
			bool isPlayer1Starting = UnityEngine.Random.value > 0.5f;
			
			if (NetworkManager.Singleton.IsHost)
			{
				FlipDecisionServerRpc(shouldFlip, isPlayer1Starting);
			}
		}
	}
\end{lstlisting}

\subsubsection{BoardManager}
A BoardManager felelős a játéktábla állapotának kezeléséért:
\begin{itemize}
	\item Tárolja a játéktábla aktuális állapotát
	\item Validálja és végrehajtja a lépéseket
	\item Ellenőrzi a játék szabályait
	\item Nyilvántartja a pontszámokat
\end{itemize}

\begin{lstlisting}[language={[Sharp]C}]
	public class BoardManager
	{
		private int[] table;
		private IPlayerController currentPlayer;
		private bool isGameOver;
		
		public void MakeMove(int index)
		{
			bool isPlayer1Turn = currentPlayer == player1;
			if (!IsMoveValid(index))
			{
				currentPlayer.RequestMove();
				return;
			}
			
			int rocksToPlace = table[index];
			table[index] = 0;
			int currentIndex = index;
			
			for (int i = 0; i < rocksToPlace; i++)
			{
				currentIndex = GetNextValidIndex(currentIndex, isPlayer1Turn);
				table[currentIndex]++;
			}
			
			HandleSpecialRules(currentIndex, isPlayer1Turn);
			
			if (!ShouldPlayerGetExtraTurn(currentIndex, isPlayer1Turn))
			{
				SwitchCurrentPlayer();
			}
		}
	}
\end{lstlisting}

\subsubsection{StoneManager}
A StoneManager a játék vizuális megjelenítéséért felelős:
\begin{itemize}
	\item Kezeli a kövek fizikai reprezentációját
	\item Animálja a kövek mozgását
	\item Szinkronizálja a vizuális megjelenítést a játék logikai állapotával
\end{itemize}

\begin{lstlisting}[language={[Sharp]C}]
	public class StoneManager : MonoBehaviour
	{
		[SerializeField] private GameObject[] rockPrefabs;
		[SerializeField] private GameObject[] rockSpawners;
		private List<List<GameObject>> rocks;
		private bool isStoneMoving = false;
		
		private IEnumerator MoveStonesCoroutine(int index, int numberOfStones, 
		bool isPlayer1Turn, bool isGameOver, int player1Score, int player2Score)
		{
			isStoneMoving = true;
			OnStoneMoveStart?.Invoke();
			
			List<GameObject> stonesToMove = CollectStonesFromPit(index, numberOfStones);
			yield return StartCoroutine(LiftStonesFromPit(stonesToMove, index));
			
			yield return StartCoroutine(DistributeStones(stonesToMove, index, isPlayer1Turn));
			
			yield return StartCoroutine(HandleCapture(lastIndex, isPlayer1Turn));
			
			isStoneMoving = false;
			OnStoneMoveComplete?.Invoke();
		}
	}
\end{lstlisting}

\section{Online játék megvalósítása}

\subsection{Hálózati architektúra}
Az online játék peer-to-peer alapú, Unity Relay szolgáltatás használatával:
\begin{itemize}
	\item A Relay szerver biztosítja a kezdeti kapcsolódást
	\item A játékosok közötti kommunikáció közvetlen
	\item A host játékos validálja a lépéseket
\end{itemize}

\subsection{Lobby rendszer}
A lobby rendszer az Unity Lobby szolgáltatására épül:
\begin{itemize}
	\item Szobák létrehozása és kezelése
	\item Játékosok csatlakozásának kezelése
	\item Szobák listázása és szűrése
\end{itemize}

\begin{lstlisting}[language={[Sharp]C}]
	public class OnlineLobby : MonoBehaviour
	{
		public static OnlineLobby Instance;
		private Lobby hostLobby;
		private Lobby joinedLobby;
		
		public async void CreateLobby(string lobbyName, bool isPrivate)
		{
			try
			{
				int maxPlayers = 2;
				string joinCode = await RelayConnection.Instance.CreateRelay();
				
				CreateLobbyOptions options = new CreateLobbyOptions
				{
					IsPrivate = isPrivate,
					Player = GetPlayer(),
					Data = new Dictionary<string, DataObject>
					{
						{ "Start", new DataObject(DataObject.VisibilityOptions.Member, joinCode) }
					}
				};
				
				Lobby lobby = await LobbyService.Instance.CreateLobbyAsync(lobbyName, maxPlayers, options);
				hostLobby = lobby;
				joinedLobby = hostLobby;
			}
			catch (LobbyServiceException e)
			{
				Debug.Log($"Failed to create lobby - {e}");
			}
		}
	}
\end{lstlisting}

\section{Mesterséges intelligencia}

Az AI játékos implementációja a következő komponensekre épül:

\subsection{AI stratégia}
Az AI játékos több szintű döntési mechanizmust használ:
\begin{itemize}
	\item Alapvető szabályok követése
	\item Lépések előnyösségének értékelése
	\item Több lépéses kombinációk keresése
\end{itemize}

\begin{lstlisting}[language={[Sharp]C}]
	public class AIPlayer : MonoBehaviour, IPlayerController
	{
		private int GetAIMove(int[] availableMoves)
		{
			if (availableMoves.Length == 0)
			{
				return -1;
			}
			
			System.Random r = new System.Random();
			int randomIndex = r.Next(0, availableMoves.Length);
			return availableMoves[randomIndex];
		}
		
		private IEnumerator SimulateThinking()
		{
			int[] availableMoves = GetAvailableMoves();
			int moveCount = availableMoves.Length;
			
			float thinkTime;
			if (moveCount >= 5) thinkTime = 2.5f;
			else if (moveCount >= 3) thinkTime = 1.7f;
			else thinkTime = 1f;
			
			yield return new WaitForSeconds(thinkTime);
			
			int aiMove = GetAIMove(availableMoves);
			IPlayerController.RaiseMoveSelected(aiMove);
		}
	}
\end{lstlisting}